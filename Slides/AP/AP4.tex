\documentclass[12pt]{beamer}
\usepackage{amsmath,tabu}
\usepackage{tikz}

\newcommand{\tm}[2]{\tikz[overlay,remember picture,baseline] \node [anchor=base] (#1) {#2};} 
\newcommand{\dvl}[3][]{% draw vertical line
  \begin{tikzpicture}[overlay,remember picture]
    \draw[#1] (#2.north) -- (#3.south);
  \end{tikzpicture}
}
\newcommand{\dhl}[3][]{% draw horizontal line
  \begin{tikzpicture}[overlay,remember picture]
    \draw[#1] (#2.east) -- (#3.west);
  \end{tikzpicture}
}
\usepackage{color}
\usepackage{fixltx2e}
\usepackage{graphicx}
\usepackage{longtable}
\usepackage{float}
\usepackage{wrapfig}
\usepackage{soul}
\usepackage{textcomp}
\usepackage{marvosym}
\usepackage{wasysym}
\usepackage{latexsym}
\usepackage{amssymb}
\usepackage{hyperref}
\tolerance=1000
\providecommand{\alert}[1]{\textbf{#1}}
\usepackage[absolute,overlay]{textpos}

\title{AP4}
\author{gar}
\date{}

\begin{document}

\maketitle

\begin{frame}
\begin{itemize}
\item Four mechanics are to be assigned four of the five available jobs such that the profit is maximized. 
The expected profit is given below. Find out which job must be declined.
\end{itemize}
\begin{center}
\begin{tabular}{|c|ccccc|}
\hline
  & 1  & 2  & 3   & 4   & 5  \\
\hline
A & 62 & 78 & 50  & 111 & 82 \\
B & 71 & 84 & 61  & 73  & 59 \\
C & 87 & 92 & 111 & 71  & 81 \\
D & 48 & 64 & 87  & 77  & 80 \\
\hline
\end{tabular}
\end{center}
\end{frame}

\begin{frame}
\begin{itemize}
\item Add a dummy mechanic
\end{itemize}
\begin{center}
\begin{tabular}{|c|ccccc|}
\hline
  & 1  & 2  & 3   & 4   & 5  \\
\hline
A & 62 & 78 & 50  & 111 & 82 \\
B & 71 & 84 & 61  & 73  & 59 \\
C & 87 & 92 & 111 & 71  & 81 \\
D & 48 & 64 & 87  & 77  & 80 \\
E & 0  & 0  & 0   & 0   & 0  \\
\hline
\end{tabular}
\end{center}
\end{frame}

\begin{frame}
\begin{itemize}
\item Convert it to a loss matrix, by subtracting each element from the maximum element in the matrix
\item Then, it gets reduced to the usual assignment problem
\end{itemize}
\begin{center}
\begin{tabular}{|c|ccccc|}
\hline
  & 1   & 2   & 3   & 4   & 5   \\
\hline
A & 49  & 33  & 61  & 0   & 29  \\
B & 40  & 27  & 50  & 38  & 52  \\
C & 24  & 19  & 0   & 40  & 30  \\
D & 63  & 47  & 24  & 34  & 31  \\
E & 111 & 111 & 111 & 111 & 111 \\
\hline
\end{tabular}
\end{center}
\end{frame}

\begin{frame}
\begin{itemize}
\item Subtract the minimum of the row from that row
\end{itemize}
\begin{center}
\begin{tabular}{|c|ccccc|}
\hline
  & 1  & 2  & 3  & 4  & 5  \\
\hline
A & 49 & 33 & 61 & 0  & 29 \\
B & 13 & 0  & 23 & 11 & 25 \\
C & 24 & 19 & 0  & 40 & 30 \\
D & 39 & 23 & 0  & 10 & 7  \\
E & 0  & 0  & 0  & 0  & 0  \\
\hline
\end{tabular}
\end{center}
\end{frame}

\begin{frame}
\begin{itemize}
\item Draw minimum number of lines to cover all zeros
\end{itemize}
\begin{center}
\begin{tabular}{|c|ccccc|}
\hline
  & 1           & 2           & 3           & 4  & 5           \\
\hline
A & \tm{a1}{49} & \tm{a5}{33} & \tm{a7}{61} & 0  & \tm{a2}{29} \\
B & 13          & 0           & 23          & 11 & 25          \\
C & 24          & 19          & 0           & 40 & 30          \\
D & 39          & 23          & 0           & 10 & 7           \\
E & \tm{a3}{0}  & \tm{a6}{0}  & \tm{a8}{0}  & 0  & \tm{a4}{0}  \\
\hline
\end{tabular}
\dhl[thick,red]{a2}{a1}
\dhl[thick,red]{a4}{a3}
\dvl[thick,red]{a5}{a6}
\dvl[thick,red]{a7}{a8}
\end{center}
\end{frame}

\begin{frame}
\begin{itemize}
\item Subtract the minimum from the uncovered elements
\end{itemize}
\begin{center}
\begin{tabular}{|c|ccccc|}
\hline
  & 1    & 2    & 3  & 4   & 5    \\
\hline
A & 49   & {40} & 68 & {0} & 29   \\
B & 6    & 0    & 23 & 5   & 18   \\
C & {17} & 19   & 0  & 33  & {23} \\
D & {32} & 23   & 0  & 3   & {0}  \\
E & 0    & {7}  & 7  & {0} & 0    \\
\hline
\end{tabular}
\end{center}
\end{frame}

\begin{frame}
\begin{itemize}
\item Draw minimum lines to cover zeros
\end{itemize}
\begin{center}
\begin{tabular}{|c|ccccc|}
\hline
  & 1           & 2           & 3  & 4          & 5           \\
\hline
A & 49          & \tm{a1}{40} & 68 & \tm{a3}{0} & 29          \\
B & 6           & 0           & 23 & 5          & 18          \\
C & \tm{a5}{17} & 19          & 0  & 33         & \tm{a6}{23} \\
D & \tm{a7}{32} & 23          & 0  & 3          & \tm{a8}{0}  \\
E & \tm{a9}{0}  & \tm{a2}{7}  & 7  & \tm{a4}{0} & \tm{a0}{0}  \\
\hline
\end{tabular}
\dhl[thick,red]{a6}{a5}
\dhl[thick,red]{a8}{a7}
\dhl[thick,red]{a0}{a9}
\dvl[thick,red]{a1}{a2}
\dvl[thick,red]{a3}{a4}
\end{center}
\end{frame}

\begin{frame}
\begin{itemize}
\item Make the assignments
\end{itemize}
\begin{center}
\begin{tabular}{|c|ccccc|}
\hline
  & 1        & 2        & 3        & 4        & 5        \\
\hline
A & 49       & {40}     & 68       & \fbox{0} & 29       \\
B & 6        & \fbox{0} & 23       & 5        & 18       \\
C & {17}     & 19       & \fbox{0} & 33       & {23}     \\
D & {32}     & 23       & 0        & 3        & \fbox{0} \\
E & \fbox{0} & {7}      & 7        & {0}      & 0        \\
\hline
\end{tabular}
\end{center}
\end{frame}
\end{document}
